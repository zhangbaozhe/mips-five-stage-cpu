% Options for packages loaded elsewhere
\PassOptionsToPackage{unicode}{hyperref}
\PassOptionsToPackage{hyphens}{url}
%
\documentclass[
]{article}
\usepackage{lmodern}
\usepackage{amssymb,amsmath}
\usepackage{ifxetex,ifluatex}
\ifnum 0\ifxetex 1\fi\ifluatex 1\fi=0 % if pdftex
  \usepackage[T1]{fontenc}
  \usepackage[utf8]{inputenc}
  \usepackage{textcomp} % provide euro and other symbols
\else % if luatex or xetex
  \usepackage{unicode-math}
  \defaultfontfeatures{Scale=MatchLowercase}
  \defaultfontfeatures[\rmfamily]{Ligatures=TeX,Scale=1}
\fi
% Use upquote if available, for straight quotes in verbatim environments
\IfFileExists{upquote.sty}{\usepackage{upquote}}{}
\IfFileExists{microtype.sty}{% use microtype if available
  \usepackage[]{microtype}
  \UseMicrotypeSet[protrusion]{basicmath} % disable protrusion for tt fonts
}{}
\makeatletter
\@ifundefined{KOMAClassName}{% if non-KOMA class
  \IfFileExists{parskip.sty}{%
    \usepackage{parskip}
  }{% else
    \setlength{\parindent}{0pt}
    \setlength{\parskip}{6pt plus 2pt minus 1pt}}
}{% if KOMA class
  \KOMAoptions{parskip=half}}
\makeatother
\usepackage{xcolor}
\IfFileExists{xurl.sty}{\usepackage{xurl}}{} % add URL line breaks if available
\IfFileExists{bookmark.sty}{\usepackage{bookmark}}{\usepackage{hyperref}}
\hypersetup{
  hidelinks,
  pdfcreator={LaTeX via pandoc}}
\urlstyle{same} % disable monospaced font for URLs
\usepackage{color}
\usepackage{fancyvrb}
\newcommand{\VerbBar}{|}
\newcommand{\VERB}{\Verb[commandchars=\\\{\}]}
\DefineVerbatimEnvironment{Highlighting}{Verbatim}{commandchars=\\\{\}}
% Add ',fontsize=\small' for more characters per line
\newenvironment{Shaded}{}{}
\newcommand{\AlertTok}[1]{\textcolor[rgb]{1.00,0.00,0.00}{\textbf{#1}}}
\newcommand{\AnnotationTok}[1]{\textcolor[rgb]{0.38,0.63,0.69}{\textbf{\textit{#1}}}}
\newcommand{\AttributeTok}[1]{\textcolor[rgb]{0.49,0.56,0.16}{#1}}
\newcommand{\BaseNTok}[1]{\textcolor[rgb]{0.25,0.63,0.44}{#1}}
\newcommand{\BuiltInTok}[1]{#1}
\newcommand{\CharTok}[1]{\textcolor[rgb]{0.25,0.44,0.63}{#1}}
\newcommand{\CommentTok}[1]{\textcolor[rgb]{0.38,0.63,0.69}{\textit{#1}}}
\newcommand{\CommentVarTok}[1]{\textcolor[rgb]{0.38,0.63,0.69}{\textbf{\textit{#1}}}}
\newcommand{\ConstantTok}[1]{\textcolor[rgb]{0.53,0.00,0.00}{#1}}
\newcommand{\ControlFlowTok}[1]{\textcolor[rgb]{0.00,0.44,0.13}{\textbf{#1}}}
\newcommand{\DataTypeTok}[1]{\textcolor[rgb]{0.56,0.13,0.00}{#1}}
\newcommand{\DecValTok}[1]{\textcolor[rgb]{0.25,0.63,0.44}{#1}}
\newcommand{\DocumentationTok}[1]{\textcolor[rgb]{0.73,0.13,0.13}{\textit{#1}}}
\newcommand{\ErrorTok}[1]{\textcolor[rgb]{1.00,0.00,0.00}{\textbf{#1}}}
\newcommand{\ExtensionTok}[1]{#1}
\newcommand{\FloatTok}[1]{\textcolor[rgb]{0.25,0.63,0.44}{#1}}
\newcommand{\FunctionTok}[1]{\textcolor[rgb]{0.02,0.16,0.49}{#1}}
\newcommand{\ImportTok}[1]{#1}
\newcommand{\InformationTok}[1]{\textcolor[rgb]{0.38,0.63,0.69}{\textbf{\textit{#1}}}}
\newcommand{\KeywordTok}[1]{\textcolor[rgb]{0.00,0.44,0.13}{\textbf{#1}}}
\newcommand{\NormalTok}[1]{#1}
\newcommand{\OperatorTok}[1]{\textcolor[rgb]{0.40,0.40,0.40}{#1}}
\newcommand{\OtherTok}[1]{\textcolor[rgb]{0.00,0.44,0.13}{#1}}
\newcommand{\PreprocessorTok}[1]{\textcolor[rgb]{0.74,0.48,0.00}{#1}}
\newcommand{\RegionMarkerTok}[1]{#1}
\newcommand{\SpecialCharTok}[1]{\textcolor[rgb]{0.25,0.44,0.63}{#1}}
\newcommand{\SpecialStringTok}[1]{\textcolor[rgb]{0.73,0.40,0.53}{#1}}
\newcommand{\StringTok}[1]{\textcolor[rgb]{0.25,0.44,0.63}{#1}}
\newcommand{\VariableTok}[1]{\textcolor[rgb]{0.10,0.09,0.49}{#1}}
\newcommand{\VerbatimStringTok}[1]{\textcolor[rgb]{0.25,0.44,0.63}{#1}}
\newcommand{\WarningTok}[1]{\textcolor[rgb]{0.38,0.63,0.69}{\textbf{\textit{#1}}}}
\usepackage{longtable,booktabs}
% Correct order of tables after \paragraph or \subparagraph
\usepackage{etoolbox}
\makeatletter
\patchcmd\longtable{\par}{\if@noskipsec\mbox{}\fi\par}{}{}
\makeatother
% Allow footnotes in longtable head/foot
\IfFileExists{footnotehyper.sty}{\usepackage{footnotehyper}}{\usepackage{footnote}}
\makesavenoteenv{longtable}
\usepackage{graphicx}
\makeatletter
\def\maxwidth{\ifdim\Gin@nat@width>\linewidth\linewidth\else\Gin@nat@width\fi}
\def\maxheight{\ifdim\Gin@nat@height>\textheight\textheight\else\Gin@nat@height\fi}
\makeatother
% Scale images if necessary, so that they will not overflow the page
% margins by default, and it is still possible to overwrite the defaults
% using explicit options in \includegraphics[width, height, ...]{}
\setkeys{Gin}{width=\maxwidth,height=\maxheight,keepaspectratio}
% Set default figure placement to htbp
\makeatletter
\def\fps@figure{htbp}
\makeatother
\setlength{\emergencystretch}{3em} % prevent overfull lines
\providecommand{\tightlist}{%
  \setlength{\itemsep}{0pt}\setlength{\parskip}{0pt}}
\setcounter{secnumdepth}{-\maxdimen} % remove section numbering
\ifluatex
  \usepackage{selnolig}  % disable illegal ligatures
\fi
\usepackage{geometry}
\usepackage{fancyhdr}
\geometry{left=2.54cm, right=2.54cm, top=2.54cm, bottom=2.54cm}
\pagestyle{fancy}
\author{Zhang Baozhe}
\date{\today}


\begin{document}

\hypertarget{header-n0}{%
\section{Report of Five-Stage Pipelined MIPS CPU}\label{header-n0}}

In this project, a basic 5-stage pipelined CPU was implemented following
the basic design described in the textbook.

\hypertarget{header-n14}{%
\section{Compilation and Test Overview}\label{header-n14}}

As described in the README file, basically to compile and test all the
sample given, type

\begin{verbatim}
> make 
> make auto -i
\end{verbatim}

This will generate a file called \texttt{DIFFERENCE} containing the
results of the \texttt{diff} command.

For more details you can refer to the README file.

\hypertarget{header-n4}{%
\section{Big Thoughts}\label{header-n4}}

I try to use an OOP style to construct the project. Since, there are
many blocks, it is natural to capsulate these bolcks in modules and
instantiate them in a topper module, i.e.,

\begin{verbatim}
top_module // in the test bench
----the instance of CPU
--------the instances of function modules in CPU
\end{verbatim}

Below is the (not that detailed) design diagram,

\begin{figure}
\centering
\includegraphics{/Users/Jack/Desktop/Screen Shot 2021-05-15 at 9.20.15 PM.png}
\caption{CPU Design}
\end{figure}

As the above diagram shows, in the project, I adopt the method of moving
the branch part from MEM stage to ID stage to speed up the performance,
which naturally generates two units, Branch Forwarding Unit and Branch
Delay Unit (though j-type instructions also go through these units).
Besides, there are also the basic forwarding and hazard detection units.
As blocked in the red rectangles, the multiplexors in these parts are
not detailed, the more detailed ones can be refered in the next section.

\hypertarget{header-n6}{%
\section{Implementation Details}\label{header-n6}}

\hypertarget{header-n12}{%
\subsection{Write and read in the register file at the same
time}\label{header-n12}}

This hazard is described as the structural hazard. It can be solved by a
forwarding inside the register file. Below is part of the implementation
of the verilog file:

\begin{Shaded}
\begin{Highlighting}[]
\KeywordTok{always}\NormalTok{ @(}\KeywordTok{posedge}\NormalTok{ CLK) }\KeywordTok{begin}
        \KeywordTok{if}\NormalTok{ (RegWrite) }\KeywordTok{begin}
\NormalTok{            registers[writeReg] \textless{}= writeData;}
        \KeywordTok{end}
    \KeywordTok{end}
    \KeywordTok{always}\NormalTok{ @(*) }\KeywordTok{begin}
        \KeywordTok{if}\NormalTok{ ((readReg1 == writeReg) \&\& RegWrite) }\CommentTok{// here}
\NormalTok{            readData1 = writeData;}
        \KeywordTok{else} 
\NormalTok{            readData1 = registers[readReg1];}
    \KeywordTok{end}
    \KeywordTok{always}\NormalTok{ @(*) }\KeywordTok{begin}
        \KeywordTok{if}\NormalTok{ ((readReg2 == writeReg) \&\& RegWrite) }\CommentTok{// and here}
\NormalTok{            readData2 = writeData;}
        \KeywordTok{else} 
\NormalTok{            readData2 = registers[readReg2];}
    \KeywordTok{end}
\end{Highlighting}
\end{Shaded}

\hypertarget{header-n59}{%
\subsection{Muxes before PC}\label{header-n59}}

Below is a diagram illustrating the details of the multiplexors before
PC:

\begin{figure}
\centering
\includegraphics{/Users/Jack/Downloads/IMG_C0107E8BA0A2-1.jpeg}
\caption{Muxex before PC}
\end{figure}

\hypertarget{header-n68}{%
\subsection{Hazard detection}\label{header-n68}}

This unit detects whether a stall will happen. It can solve the
following types of hazard:

\begin{verbatim}
add $1, $2, $3
# at least 1-cycle stall
beq $1, $2, address 
----------------------------
lw $1, 20($2)
# at least 2-cycle stall
beq $1, $2, address
----------------------------
addi $1, $zero, 20
# at least 1-cycle stall
jr $1
----------------------------
lw $1, 20($2)
# at least 2-cycle stall
jr $1
\end{verbatim}

The detailed logic can be referred from the source code.

\hypertarget{header-n134}{%
\subsection{ALU src details}\label{header-n134}}

This part is the most messy part in the CPU, since there are many
different sources for ALU and the control unit has to decide to use
which source. There are basically three types of souces:
arithematic(from registers or immediates), load/store address
calculation, and shift. The details can be referred from the
\texttt{CPU.v} file.

\hypertarget{header-n73}{%
\subsection{Forwarding details}\label{header-n73}}

This part basically implementates the forwarding scheme for ALU. As
described in the textbook, it solves the regular arithmetic types:

\begin{itemize}
\item
  1a. EX/MEM.RegisterRd = ID/EX.RegisterRs
\end{itemize}

\begin{itemize}
\item
  1b. EX/MEM.RegisterRd = ID/EX.RegisterRt
\item
  2a. MEM/WB.RegisterRd = ID/EX.RegisterRs
\item
  2b. MEM/WB.RegisterRd = ID/EX.RegisterRt
\end{itemize}

Besides, it also supports the forwarding for store word, e.g.,

\begin{verbatim}
add $1, $2, $3
# forwarding
sw $1, 20($4)
\end{verbatim}

\hypertarget{header-n126}{%
\subsection{Branch forward and related details}\label{header-n126}}

This unit is similar to the forwarding unit in the EX stage, but it has
some to perform after the stall of the branch or jump instructions.

\hypertarget{header-n66}{%
\section{Performance Analysis}\label{header-n66}}

\begin{longtable}[]{@{}llll@{}}
\toprule
& Number of Instructions & Number of Clock Cycles & CPI\tabularnewline
\midrule
\endhead
Test 1 & 53 & 56 & 1.057\tabularnewline
Test 2 & 12 & 15 & 1.250\tabularnewline
Test 3 & 14 & 18 & 1.286\tabularnewline
Test 4 & 14 & 17 & 1.214\tabularnewline
Test 5 & 25 & 179 & 7.160\tabularnewline
Test 6 & 52 & 54 & 1.038\tabularnewline
Test 7 & 45 & 47 & 1.044\tabularnewline
Test 8 & 25 & 31 & 1.240\tabularnewline
\bottomrule
\end{longtable}

\hypertarget{header-n34}{%
\section{Pitfalls}\label{header-n34}}

\hypertarget{header-n10}{%
\subsection{The given file can be misleading}\label{header-n10}}

The misleading part, I think during my debugging, is that the reading of
one instruction or some data in the memory needs the CLK to trigger
which is conunter-intuitive. Thus, to improve the performance, I change
the reading of the RAM to asychronously triggered.

\hypertarget{header-n37}{%
\subsection{The branch/jump delay slot needs
handling}\label{header-n37}}

By default, when taking a branch or jump instruction (whether the branch
is taken or not) the next instruction is automatically loaded into the
IF stage after the branch goes to ID stage because at that time the
branch is not taken and \texttt{PC+4} just takes place. This is actually
a good thing for the compilers to effciently use these slots, but in
this project, it needs handling properly when the branch or jump is
indeed taken.

The solution is simple. For jump and the taken branch instructions, the
program will detect: 1. whether the branch/jump will be taken; 2. if
taken, then delay unit will flush the ID/EX register.

\hypertarget{header-n43}{%
\subsection{\texorpdfstring{The forwarding scheme of JAL and JR
}{The forwarding scheme of JAL and JR }}\label{header-n43}}

Basically, \texttt{jal} is similar to plain \texttt{j} except for the
link action, and it can be reduced to

\begin{verbatim}
jal 20
----------------------------
j 20
addi $at, $zero, PC+4
\end{verbatim}

Thus, only use some multiplexors to control the data flow of the
registers and control the control unit to generate control code can
achieve this instruction.

The \texttt{jr} is a R-type instruction which is different from the
\texttt{j} or \texttt{jal}. To achieve this instruction, I add it to the
branch part(in my design, \texttt{beq} has a branch code of
\texttt{2\textquotesingle{}b01}, \texttt{bne} has a branch code of
\texttt{2\textquotesingle{}b10}, not branch for
\texttt{2\textquotesingle{}b0}, and \texttt{jr} for
\texttt{2\textquotesingle{}b11}).

\end{document}
